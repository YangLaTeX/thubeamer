% \iffalse meta-comment
% !TeX program  = XeLaTeX
% !TeX encoding = UTF-8
%
% Copyright (C) 2019--2021 by Jingxuan Yang <yanglatex2e@gmail.com>
%
% This file may be distributed and/or modified under the
% conditions of the LaTeX Project Public License, either version 1.3c
% of this license or (at your option) any later version.
% The latest version of this license is in:
%
% http://www.latex-project.org/lppl.txt
%
% and version 1.3c or later is part of all distributions of LaTeX
% version 2008/05/04 or later.

%<*internal>
\def\nameofplainTeX{plain}
\ifx\fmtname\nameofplainTeX\else
  \expandafter\begingroup
\fi
%</internal>
%<*install>

\input docstrip
\keepsilent
\askforoverwritefalse
\preamble

This is a generated file.

Copyright (C) 2019--\the\year by Jingxuan Yang <yanglatex2e@gmail.com>

This file may be distributed and/or modified under the
conditions of the LaTeX Project Public License, either version 1.3c
of this license or (at your option) any later version.
The latest version of this license is in:

http://www.latex-project.org/lppl.txt

and version 1.3c or later is part of all distributions of LaTeX
version 2008/05/04 or later.

To produce the documentation run the original source files ending with `.dtx'
through LaTeX.

\endpreamble
\postamble

This package consists of the file  thubeamer.dtx,
             and the derived files thubeamer.pdf,
                                   beamerthemethubeamer.sty,
                                   beamercolorthemethubeamer.sty,
                                   beamerinnerthemethubeamer.sty,
                                   beamerouterthemethubeamer.sty.

\endpostamble

\usedir{tex/latex/thubeamer}
\generate
  {
    \file{dtx-style.sty}{\from{\jobname.dtx}{dtx-style}}
  \file{beamerthemethubeamer.sty}{\from{\jobname.dtx}{thubeamer}}
	\file{beamercolorthemethubeamer.sty}{\from{\jobname.dtx}{thucolorstyle}}
	\file{beamerinnerthemethubeamer.sty}{\from{\jobname.dtx}{thuinnerstyle}}
	\file{beamerouterthemethubeamer.sty}{\from{\jobname.dtx}{thuouterstyle}}
  }

\Msg{***********************************************************}
\Msg{*}
\Msg{* To finish the installation you have to move the following}
\Msg{* files into a directory searched by TeX:}
\Msg{*}
\Msg{* The recommended directory is TEXMF/tex/latex/thubeamer}
\Msg{*}
\Msg{* \space\space dtx-style.sty}
\Msg{* \space\space beamerthemethubeamer.sty}
\Msg{* \space\space beamercolorthemethubeamer.sty}
\Msg{* \space\space beamerinnerthemethubeamer.sty}
\Msg{* \space\space beamerouterthemethubeamer.sty}
\Msg{*}
\Msg{* To produce the documentation run the files ending with}
\Msg{* `.dtx' through LaTeX.}
\Msg{*}
\Msg{* Happy TeXing!}
\Msg{***********************************************************}

%</install>
%<install>\endbatchfile
%<*internal>

\usedir{tex/latex/thubeamer}
\generate
  {
	\file{\jobname.ins}{\from{\jobname.dtx}{install}}
  }
\nopreamble\nopostamble

\ifx\fmtname\nameofplainTeX
  \expandafter\endbatchfile
\else
  \expandafter\endgroup
\fi
%</internal>
%
%<thubeamer|thucolorstyle|thuinnerstyle|thuouterstyle>\NeedsTeXFormat{LaTeX2e}
%<thubeamer>\ProvidesPackage{beamerthemethubeamer}
%<thucolorstyle>\ProvidesPackage{beamercolorthemethubeamer}
%<thuinnerstyle>\ProvidesPackage{beamerinnerthemethubeamer}
%<thuouterstyle>\ProvidesPackage{beamerouterthemethubeamer}
%<thubeamer|thucolorstyle|thuinnerstyle|thuouterstyle>[2021/11/15 v1.1.0 A Beamer Theme for Tsinghua University (THU)]
%
%<*driver>
\ProvidesFile{thubeamer.dtx}[2021/11/15 v1.1.0 A Beamer Theme for Tsinghua University (THU)]
\documentclass{ltxdoc}
\usepackage{dtx-style}

\EnableCrossrefs
\CodelineIndex
\RecordChanges

\begin{document}
  \DocInput{\jobname.dtx}
\end{document}
%</driver>
%
% \fi
%
% \CheckSum{0}
% \CharacterTable
%  {Upper-case    \A\B\C\D\E\F\G\H\I\J\K\L\M\N\O\P\Q\R\S\T\U\V\W\X\Y\Z
%   Lower-case    \a\b\c\d\e\f\g\h\i\j\k\l\m\n\o\p\q\r\s\t\u\v\w\x\y\z
%   Digits        \0\1\2\3\4\5\6\7\8\9
%   Exclamation   \!     Double quote  \"     Hash (number) \#
%   Dollar        \$     Percent       \%     Ampersand     \&
%   Acute accent  \'     Left paren    \(     Right paren   \)
%   Asterisk      \*     Plus          \+     Comma         \,
%   Minus         \-     Point         \.     Solidus       \/
%   Colon         \:     Semicolon     \;     Less than     \<
%   Equals        \=     Greater than  \>     Question mark \?
%   Commercial at \@     Left bracket  \[     Backslash     \\
%   Right bracket \]     Circumflex    \^     Underscore    \_
%   Grave accent  \`     Left brace    \{     Vertical bar  \|
%   Right brace   \}     Tilde         \~}
%
% \GetFileInfo{\jobname.dtx}
%
% \changes{v1.0.0}{2021/07/07}{Initially design thubeamer}
%
% \DoNotIndex{\begin,\end,\begingroup,\endgroup}
% \DoNotIndex{\ifx,\ifdim,\ifnum,\ifcase,\else,\or,\fi,\-}
% \DoNotIndex{\let,\def,\xdef,\newcommand,\renewcommand}
% \DoNotIndex{\expandafter,\csname,\endcsname,\relax,\protect}
% \DoNotIndex{\Huge,\huge,\LARGE,\Large,\large,\normalsize}
% \DoNotIndex{\small,\footnotesize,\scriptsize,\tiny}
% \DoNotIndex{\normalfont,\bfseries,\slshape,\interlinepenalty}
% \DoNotIndex{\hfil,\par,\hskip,\vskip,\vspace,\quad}
% \DoNotIndex{\centering,\raggedright}
% \DoNotIndex{\c@secnumdepth,\@startsection,\@setfontsize}
% \DoNotIndex{\ ,\@plus,\@minus,\p@,\z@,\@m,\@M,\@ne,\m@ne}
% \DoNotIndex{\@@par,\DeclareOperation,\RequirePackage,\LoadClass}
% \DoNotIndex{\AtBeginDocument,\AtEndDocument}
%
% \IndexPrologue{\section*{索~~~~引}}
% \GlossaryPrologue{\section*{修改记录}}
%
% \definecolor{beamer@headercolor}{RGB}{129,48,140}
% \title{\bfseries\color{beamer@headercolor}\thubeamer:清华大学风格\\ Beamer主题}
% \author{{\Large\fangsong 杨敬轩}\\[5pt]\texttt{yanglatex2e@gmail.com}\\[5pt]\texttt{yangjx20@mails.tsinghua.edu.cn}}
% \date{\fileversion\ (\filedate)}
%
% \maketitle\thispagestyle{empty}
%
% \vskip 0.5cm
%
% \def\abstractname{\Large 摘\quad 要}
% \begin{abstract}\normalsize\vskip0.5cm
% \thubeamer~主题为作者在准备课程展示时设计制作,
% 旨在帮助清华大学师生进行学术交流或其他需要演示文稿活动时,方便的使用 \LaTeX{} 制作含有学校特色的演示文稿。
% \end{abstract}
%
% \vspace*{2em}
% \def\abstractname{\Large 免责声明}
% \begin{abstract}
% \noindent\normalsize
%  \begin{enumerate}
% \item 本模板的发布遵守 \href{http://www.latex-project.org/lppl.txt}{\LaTeX\ Project Public License 1.3c} 以及其后的最新版本,使用前请认真阅读协议内
%   容。
%  \item 本主题为作者个人制作,使用仅供参考,任何由于使用本主题而引起的任何问题均与本主题作者无关。
%  \item 任何个人或组织以本模板为基础进行修改、扩展而生成的新的专用模板,请严格遵
%   守 \href{http://www.latex-project.org/lppl.txt}{\LaTeX\ Project Public License 1.3c} 协议以及其后的最新版本。由于违犯协议而引起的任何纠纷争端均与
%   本模板作者无关。
%  \end{enumerate}
% \end{abstract}
%
%
% \clearpage
% \pagestyle{fancy}
% \begin{multicols}{2}[
%   \setlength{\columnseprule}{.4pt}
%   \setlength{\columnsep}{18pt}]
%   \tableofcontents
% \end{multicols}
% \clearpage
%
% \section{模板介绍}
% \thubeamer\ (\textbf{T}sing\textbf{H}ua \textbf{U}niversity \LaTeX{}
% \textbf{Beamer} Template) 是为了帮助清华大学师生撰写演示文稿而编写的 \LaTeX{} Beamer 模板。
%
% 本文档将尽量完整的介绍模板的使用方法,如有不清楚之处可以参考示例文档或者根据
% 第~\ref{sec:howtoask} 节说明提问,有兴趣者可以联系作者参与完善此手册,非常欢迎你清学子对本代码作出贡献。
%
% \section{安装}
% \label{sec:installation}
%
% \thubeamer\ 已经上传 CTAN,已包含在 TeXLive 与 MiKTeX 发行版中。
% 安装方法:打开命令行,输入以下命令即可
% \begin{shell}
% $ tlmgr install thubeamer
% \end{shell}
% 阅读本说明文档可以使用以下命令:
% \begin{shell}
% $ texdoc thubeamer
% \end{shell}
%
% 如果要使用开发版,需自己下载,\thubeamer\ 相关链接:
% \begin{itemize}
% \item 主页:\href{https://github.com/YangLaTeX/thubeamer}{GitHub}
% \item 下载:\href{http://www.ctan.org/pkg/thubeamer}{CTAN}
% \end{itemize}
%
% \subsection{模板的组成}
% 下表列出了 \thubeamer{} 的主要文件及其功能介绍:
%
% \begin{longtable}{l|p{8cm}}
% \toprule
% {\heiti 文件(夹)} & {\heiti 功能描述}\\\midrule
% \endfirsthead
% \midrule
% {\heiti 文件(夹)} & {\heiti 功能描述}\\\midrule
% \endhead
% \endfoot
% \endlastfoot
% thubeamer.dtx & 主题宏包、说明文档以及主题驱动文件的混排文件\\
% main.tex & 主题测试文档\\
% figures/ & 主题相关矢量图存储文件夹\\
% makedoc.bat & 生成主题与用户手册脚本\\
% makebeamer.bat & 编译测试样例脚本\\
% makeclean.bat & 清理编译过程中间文件脚本\\
% makecleanall.bat & 清理编译过程中间文件与样式文件脚本\\
% Makefile & Linux \& Mac OS脚本\\
% \textbf{thubeamer.pdf} & 用户手册(本文档)\\
% \bottomrule
% \end{longtable}
% \textbf{注:}使用前,请仔细阅读\textbf{thubeamer.pdf},即本文档。
%
% \section{使用说明}
% \label{sec:usage}
% 本手册假定用户已经能处理一般的 \LaTeX\ 文档,并对 \BibTeX\ 有一定了解。如果
% 从来没有接触过 \TeX\ 和 \LaTeX,建议先学习相关的基础知识。
%
% \subsection{关于提问}
% \label{sec:howtoask}
% 按照优先级推荐提问的位置如下:
%
% \begin{itemize}
% \item Github Issues: \href{http://github.com/YangLaTeX/thubeamer/issues}{http://github.com/YangLaTeX/thubeamer/issues}
% \item Email: \href{mailto:yanglatex2e@gmail.com}{yanglatex2e@gmail.com}, \href{mailto:yangjx20@mails.tsinghua.edu.cn}{yangjx20@mails.tsinghua.edu.cn}
% \end{itemize}
%
% \subsection{示例文件}
% \label{sec:userguide}
% 模板核心文件有四个\file{*.sty},但是如果
% 没有示例文档用户会发现很难下手。所以推荐新用户从模板自带的示例文档入手,里面包
% 括了Beamer写作用到的所有命令及其使用方法,只需要用自己的内容进行相应替换就可以。
% 对于不清楚的命令可以查阅本手册。下面的例子描述了模板中章节的组织形式,来自于示
% 例文档,具体内容可以参考模板附带的 \file{main.tex}。
%
% \lstinputlisting[style=lstStyleLaTeX]{main.tex}
%
% \subsection{论文选项}
% \label{sec:option}
%
% 论文选项在\file{main.tex} 文件的开头描述,此处不赘述。
%
% \section{使用模板}
% \subsection{如何编译}\label{SubSec:HowToMake}
% \subsubsection{Windows用户}
% 主题项目里已经列出了脚本:
% 首先双击\file{makedoc.bat} 脚本,生成主题和用户手册;
% 然后双击\file{makebeamer.bat} 脚本生成测试 Beamer 文件预览;
% 最后双击\file{makeclean.bat} 清理编译过程的中间文件;
% 双击\file{makecleanall.bat}可以清除辅助文件以及样式文件。
%
% \subsubsection{Linux \& Mac OS用户}
% 首先运行下面命令生成主题和用户手册;
% \begin{shell}
% $ make doc
% \end{shell}
%
% 然后运行下面命令生成测试 Beamer 文件预览;
% \begin{shell}
% $ make beamer
% \end{shell}
%
% 最后运行下面命令清理编译过程的中间文件。
% \begin{shell}
% $ make clean
% \end{shell}
%
% 若运行下面命令则清理编译过程的中间文件以及所有样式文件。
% \begin{shell}
% $ make cleanall
% \end{shell}
%
% \section{致谢}
% \label{sec:thanks}
% 感谢 \href{https://github.com/StickCui/XDUstyle-Beamer-Theme}{XDUstyle}、\href{https://github.com/tl3shi/THUBeamer}{thubeamer} 模板的作者,本模板基于他们改编而来!
%
% 欢迎各位到 \href{http://github.com/YangLaTeX/thubeamer/}{\thubeamer\ Github 主页}贡献!
%
% \StopEventually{\PrintChanges\PrintIndex}
% \clearpage
%
% \section{实现细节}
%
% \subsection{主题主文件部分}
% 导入必要宏包
%    \begin{macrocode}
%<*thubeamer>
\def\thubeamer{\textsc{thu}\-\textsc{Beamer}}
\RequirePackage{tikz}
\usetikzlibrary{external}
\RequirePackage{pgf}
\RequirePackage{multicol}
\RequirePackage{multimedia}
\RequirePackage{calc}
\RequirePackage{amsmath}
\RequirePackage{amsthm}
\RequirePackage{amssymb}
\RequirePackage{bm}
\RequirePackage{graphicx}
\RequirePackage{tabularx}
\RequirePackage{booktabs}
\RequirePackage{multirow}
\RequirePackage{enumerate}
\RequirePackage{hyperref}
\RequirePackage{algorithm}
\RequirePackage{algorithmic}
\RequirePackage[T1]{fontenc}
\RequirePackage{latexsym,xcolor,multicol,calligra}
\RequirePackage{pstricks,listings,stackengine}
\RequirePackage[sort&compress,numbers]{natbib}
%    \end{macrocode}
%
% 主题选项
%    \begin{macrocode}
\DeclareOptionBeamer{thupurple}{
  \PassOptionsToPackage{thupurple}{beamercolorthemethubeamer}
  \PassOptionsToPackage{thupurple}{beamerinnerthemethubeamer}
  \PassOptionsToPackage{thupurple}{beamerouterthemethubeamer}
}
\DeclareOptionBeamer{thupurple2}{
  \PassOptionsToPackage{thupurple2}{beamercolorthemethubeamer}
  \PassOptionsToPackage{thupurple2}{beamerinnerthemethubeamer}
  \PassOptionsToPackage{thupurple2}{beamerouterthemethubeamer}
}
\DeclareOptionBeamer{smoothbars}{
  \PassOptionsToPackage{smoothbars}{beamercolorthemethubeamer}
  \PassOptionsToPackage{smoothbars}{beamerouterthemethubeamer}
}
\DeclareOptionBeamer{sidebar}{
  \PassOptionsToPackage{sidebar}{beamercolorthemethubeamer}
  \PassOptionsToPackage{sidebar}{beamerouterthemethubeamer}
}
\DeclareOptionBeamer{sectiontoc}{
  \PassOptionsToPackage{sectiontoc}{beamerinnerthemethubeamer}
}
\DeclareOptionBeamer{subsectiontoc}{
  \PassOptionsToPackage{subsectiontoc}{beamerinnerthemethubeamer}
}
\DeclareOptionBeamer{en}{
  \PassOptionsToPackage{en}{beamerinnerthemethubeamer}
}
\ProcessOptionsBeamer
%    \end{macrocode}
%
% 特殊设置
%    \begin{macrocode}
\mode<presentation>
{
  \useinnertheme{thubeamer}
  \useoutertheme{thubeamer}
  \usecolortheme{thubeamer}
}

\mode<all>
%</thubeamer>
%    \end{macrocode}
% \subsection{主题配色文件部分}
%
% 参数设置
%    \begin{macrocode}
%<*thucolorstyle>
\DeclareOptionBeamer{thupurple}{\def\beamer@thucolor{thupurple}}
\DeclareOptionBeamer{thupurple2}{\def\beamer@thucolor{thupurple2}}
\DeclareOptionBeamer{smoothbars}{\def\beamer@thubar{smoothbars}}
\DeclareOptionBeamer{sidebar}{\def\beamer@thubar{sidebar}}
\ProcessOptionsBeamer

\def\beamer@thucolorpurple{thupurple}
\def\beamer@thusidebar{sidebar}
%    \end{macrocode}
%
% 颜色设置。
%    \begin{macrocode}
\mode<presentation>

\ifx\beamer@thucolor\beamer@thucolorpurple
  \definecolor{beamer@textcolor}{RGB}{119,33,151}
  \definecolor{beamer@headercolor}{RGB}{129,48,140} % purple
  \ifx\beamer@thubar\beamer@thusidebar
    \definecolor{beamer@sidebarcolor}{RGB}{129,48,140}
  \else
    \definecolor{beamer@sidebarcolor}{RGB}{129,48,140}
  \fi
\else
  \definecolor{beamer@textcolor}{RGB}{119,33,151}
  \definecolor{beamer@headercolor}{RGB}{129,48,140}
  \ifx\beamer@thubar\beamer@thusidebar
    \definecolor{beamer@sidebarcolor}{RGB}{129,48,140}
  \else
    \definecolor{beamer@sidebarcolor}{RGB}{129,48,140}
  \fi
\fi

% \setbeamercolor{footline}{bg=beamer@headercolor}
\setbeamercolor{title}{fg=white,bg=beamer@headercolor}
\ifx\beamer@thubar\beamer@thusidebar
  \setbeamercolor{frametitle}{fg=beamer@headercolor,bg=beamer@headercolor!10}
\else
  \setbeamercolor{frametitle}{fg=white,bg=beamer@headercolor}
\fi
\setbeamerfont{frametitle}{size=\large}

\setbeamercolor{secondbottomline}{fg=white,bg=beamer@headercolor!50!black}
\setbeamercolor{firstbottomline}{fg=white,bg=beamer@headercolor}

\setbeamercolor{palette primary}{use=structure,fg=white,bg=structure.fg}
\setbeamercolor{palette secondary}{use=structure,fg=white,bg=structure.fg!75!black}
\setbeamercolor{palette tertiary}{use=structure,fg=white,bg=structure.fg!50!black}
\setbeamercolor{palette quaternary}{fg=white,bg=structure.fg!50!black}

\setbeamercolor{block title}{fg=white,bg=beamer@headercolor}
\setbeamercolor{block body}{bg=beamer@headercolor!10}
\setbeamercolor{block title example}{%
  use={normal text,example text},fg=white,bg=example text.fg!75!green
}
\setbeamercolor{fine separation line}{}
\setbeamercolor{item projected}{fg=white}

\setbeamercolor{palette sidebar primary}{use=normal text,fg=normal text.fg}
\setbeamercolor{palette sidebar secondary}{use=structure,fg=structure.fg}
\setbeamercolor{palette sidebar tertiary}{use=normal text,fg=normal text.fg}
\setbeamercolor{palette sidebar quaternary}{use=structure,fg=structure.fg}

\setbeamercolor{section in sidebar}{fg=beamer@textcolor}
\setbeamercolor{section in sidebar shaded}{fg=gray}
\setbeamercolor{separation line}{}
\ifx\beamer@thubar\beamer@thusidebar
  \setbeamercolor{sidebar}{bg=beamer@sidebarcolor!10}
\else
  \setbeamercolor{sidebar}{bg=beamer@sidebarcolor}
\fi
\setbeamercolor{sidebar}{parent=palette primary}
\setbeamercolor{structure}{fg=beamer@headercolor}
\setbeamercolor{subsection in sidebar}{fg=beamer@textcolor}
\setbeamercolor{subsection in sidebar shaded}{fg=gray}
\setbeamercolor{logo}{bg=beamer@headercolor!20}

\mode<all>
%</thucolorstyle>
%    \end{macrocode}
%
% \subsection{内部主题文件部分}
%
% 必要宏包
%    \begin{macrocode}
%<*thuinnerstyle>
\RequirePackage{tikz}
\usetikzlibrary{external}
%    \end{macrocode}
%
% 参数设置
% \changes{v1.1.0}{2021/11/15}{add option `en' for preparing beamer in English}
%    \begin{macrocode}
\DeclareOptionBeamer{en}{\def\beamer@thulang{english}}
\DeclareOptionBeamer{sectiontoc}{\def\beamer@thusectoc{sectiontoc}}
\DeclareOptionBeamer{subsectiontoc}{\def\beamer@thusubsectoc{subsectiontoc}}
\ProcessOptionsBeamer

\def\beamer@thulanguage{english}
\def\beamer@thusectiontoc{sectiontoc}
\def\beamer@thusubsectiontoc{subsectiontoc}
%    \end{macrocode}
%
% 特殊设置
%    \begin{macrocode}
\mode<presentation>

\ifx\beamer@thulang\beamer@thulanguage
  \relax
\else
  \RequirePackage{ctex}
\fi

\useinnertheme{circles}
% \usefonttheme{structurebold}
\setbeamertemplate{caption}[numbered]{}
\useinnertheme[shadow]{rounded}

\newcommand\varparallel{%
  \mathrel{%
    \text{%
      \tikz[baseline] \draw (0em,-0.3ex) -- (.4em,1.7ex) (.2em,-0.3ex) -- (.6em,1.7ex);
    }%
  }%
}

\ifx\beamer@thusectoc\beamer@thusectiontoc
  \AtBeginSection[]{
    \begin{frame}
      \begin{multicols}{2}
        \tableofcontents[sectionstyle=show/shaded,subsectionstyle=show/shaded/hide,subsubsectionstyle=show/shaded/hide]
      \end{multicols}
    \end{frame}
  }
\fi

\ifx\beamer@thusubsectoc\beamer@thusubsectiontoc
  \AtBeginSubsection[]{
    \begin{frame}
      \begin{multicols}{2}
        \tableofcontents[sectionstyle=show/shaded,subsectionstyle=show/shaded/hide,subsubsectionstyle=show/shaded/hide]
      \end{multicols}
    \end{frame}
  }
\fi

\mode<all>
%</thuinnerstyle>
%    \end{macrocode}
% \subsection{外部主题文件部分}
%
%    \begin{macrocode}
%<*thuouterstyle>
%    \end{macrocode}
%
% 参数设置
%    \begin{macrocode}
\DeclareOptionBeamer{smoothbars}{\def\beamer@thubar{smoothbars}}
\DeclareOptionBeamer{sidebar}{\def\beamer@thubar{sidebar}}
\ProcessOptionsBeamer
\def\beamer@thusidebar{sidebar}

\mode<presentation>

% 取消导航符号
\setbeamertemplate{navigation symbols}{}

\ifx\beamer@thubar\beamer@thusidebar
  \useoutertheme[width=0.17\linewidth]{sidebar}
  \logo{\includegraphics[width=0.08\linewidth]{thulogo.pdf}}
\else
  \useoutertheme[footline=authorinstitutetitle,subsection=false]{smoothbars}
\fi

% set footline
\makeatletter
\newcommand{\frameofframes}{/}
\newcommand{\setframeofframes}[1]{
  \renewcommand{\frameofframes}{#1}
}
\setbeamertemplate{footline}{%
  \begin{beamercolorbox}[colsep=1.5pt]{upper separation line foot}
  \end{beamercolorbox}
  \begin{beamercolorbox}[ht=2.5ex,dp=1.125ex,%
    leftskip=.3cm,rightskip=.3cm plus1fil]{firstbottomline}%
    \leavevmode{\usebeamerfont{author in head/foot}\insertshortauthor}%
    \hfill%
    {\usebeamerfont{institute in head/foot}%
    \usebeamercolor[fg]{frame number}%
      \insertshortinstitute
    }%
  \end{beamercolorbox}%
  \begin{beamercolorbox}[ht=2.5ex,dp=1.125ex,%
    leftskip=.3cm,rightskip=.3cm plus1fil]{secondbottomline}%
    {\usebeamerfont{title in head/foot}\insertshorttitle}%
    \hfill%
    {\usebeamerfont{frame number}\usebeamercolor[fg]{frame number}\insertframenumber~\frameofframes~\inserttotalframenumber}
  \end{beamercolorbox}%
  \begin{beamercolorbox}[colsep=1.5pt]{lower separation line foot}
  \end{beamercolorbox}
}
\makeatother

\mode<all>
%</thuouterstyle>
%    \end{macrocode}
%
% \iffalse
%    \begin{macrocode}
%<*dtx-style>
\ProvidesPackage{dtx-style}
\RequirePackage{hypdoc}
\RequirePackage[UTF8,scheme=chinese,fontset=windowsnew]{ctex}
\RequirePackage{newpxtext}
\RequirePackage{newpxmath}
\RequirePackage[top=2.5cm, bottom=2.5cm, left=3cm, right=2cm, headsep=8mm]{geometry}
\RequirePackage{array,longtable,booktabs}
\RequirePackage{listings}
\RequirePackage{fancyhdr}
\RequirePackage{xcolor}
\RequirePackage{enumitem}
\RequirePackage{etoolbox}
\RequirePackage{metalogo}
\RequirePackage{hyperref}

\colorlet{thu@macro}{purple!60!black}
\colorlet{thu@env}{purple!70!black}
\colorlet{thu@option}{purple}
\patchcmd{\PrintMacroName}{\MacroFont}{\MacroFont\bfseries\color{thu@macro}}{}{}
\patchcmd{\PrintDescribeMacro}{\MacroFont}{\MacroFont\bfseries\color{thu@macro}}{}{}
\patchcmd{\PrintDescribeEnv}{\MacroFont}{\MacroFont\bfseries\color{thu@env}}{}{}
\patchcmd{\PrintEnvName}{\MacroFont}{\MacroFont\bfseries\color{thu@env}}{}{}

\def\DescribeOption{%
  \leavevmode\@bsphack\begingroup\MakePrivateLetters%
  \Describe@Option}
\def\Describe@Option#1{\endgroup
  \marginpar{\raggedleft\PrintDescribeOption{#1}}%
  \thu@special@index{option}{#1}\@esphack\ignorespaces}
\def\PrintDescribeOption#1{\strut \MacroFont\bfseries\sffamily\color{thu@option} #1\ }
\def\thu@special@index#1#2{\@bsphack
  \begingroup
    \HD@target
    \let\HDorg@encapchar\encapchar
    \edef\encapchar usage{%
      \HDorg@encapchar hdclindex{\the\c@HD@hypercount}{usage}%
    }%
    \index{#2\actualchar{\string\ttfamily\space#2}
           (#1)\encapchar usage}%
    \index{#1:\levelchar#2\actualchar
           {\string\ttfamily\space#2}\encapchar usage}%
  \endgroup
  \@esphack}

\lstdefinestyle{lstStyleBase}{%
   basicstyle=\small\ttfamily,
   aboveskip=\medskipamount,
   belowskip=\medskipamount,
   lineskip=0pt,
   boxpos=c,
   showlines=false,
   extendedchars=true,
   upquote=true,
   tabsize=2,
   showtabs=false,
   showspaces=false,
   showstringspaces=false,
   numbers=none,
   linewidth=\linewidth,
   xleftmargin=4pt,
   xrightmargin=0pt,
   resetmargins=false,
   breaklines=true,
   breakatwhitespace=false,
   breakindent=0pt,
   breakautoindent=true,
   columns=flexible,
   keepspaces=true,
   gobble=2,
   framesep=3pt,
   rulesep=1pt,
   framerule=1pt,
   backgroundcolor=\color{gray!5},
   stringstyle=\color{purple!40!black!100},
   keywordstyle=\bfseries\color{purple!50!black},
   commentstyle=\slshape\color{black!60}}

\lstdefinestyle{lstStyleShell}{%
   style=lstStyleBase,
   frame=l,
   rulecolor=\color{purple},
   language=bash}

\definecolor{beamer@headercolor}{RGB}{21,95,130}
\lstdefinestyle{lstStyleLaTeX}{%
   style=lstStyleBase,
   frame=l,
   rulecolor=\color{beamer@headercolor},
   language=[LaTeX]TeX}

\lstnewenvironment{latex}{\lstset{style=lstStyleLaTeX}}{}
\lstnewenvironment{shell}{\lstset{style=lstStyleShell}}{}

\setlist{nosep}

\DeclareDocumentCommand{\option}{m}{\textsf{#1}}
\DeclareDocumentCommand{\env}{m}{\texttt{#1}}
\DeclareDocumentCommand{\pkg}{s m}{%
  \texttt{#2}\IfBooleanF#1{\thu@special@index{package}{#2}}}
\DeclareDocumentCommand{\file}{s m}{%
  \texttt{#2}\IfBooleanF#1{\thu@special@index{file}{#2}}}
\newcommand{\myentry}[1]{%
  \marginpar{\small\raggedleft\color{purple}\bfseries\strut #1}}
\newcommand{\note}[2][Note]{{%
  \color{magenta}{\bfseries #1}\emph{#2}}}
\def\thubeamer{\textsc{Thu}\-\textsc{Beamer}}
\def\thu{清华大学}
%</dtx-style>
%    \end{macrocode}
% \fi
%
% \Finale
%
\endinput
% \iffalse
%  Local Variables:
%  mode: doctex
%  TeX-master: t
%  End:
% \fi
